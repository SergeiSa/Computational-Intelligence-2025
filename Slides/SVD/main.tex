\documentclass{beamer}

\pdfmapfile{+sansmathaccent.map}


\mode<presentation>
{
	\usetheme{Warsaw} % or try Darmstadt, Madrid, Warsaw, Rochester, CambridgeUS, ...
	\usecolortheme{seahorse} % or try seahorse, beaver, crane, wolverine, ...
	\usefonttheme{serif}  % or try serif, structurebold, ...
	\setbeamertemplate{navigation symbols}{}
	\setbeamertemplate{caption}[numbered]
} 


%%%%%%%%%%%%%%%%%%%%%%%%%%%%
% itemize settings

\definecolor{mypaleblue}{RGB}{240, 240, 255}
\definecolor{mylightblue}{RGB}{120, 150, 255}
\definecolor{myblue}{RGB}{90, 90, 255}
\definecolor{mygblue}{RGB}{70, 110, 240}
\definecolor{mydarkblue}{RGB}{0, 0, 180}
\definecolor{myblackblue}{RGB}{40, 40, 120}

\definecolor{mygreen}{RGB}{0, 200, 0}
\definecolor{mydarkgreen}{RGB}{0, 120, 0}
\definecolor{mygreen2}{RGB}{245, 255, 230}

\definecolor{mygray}{gray}{0.8}
\definecolor{mygray2}{RGB}{130, 130, 130}
\definecolor{mydarkgray}{RGB}{80, 80, 160}
\definecolor{mylightgray}{RGB}{160, 160, 160}

\definecolor{mydarkred}{RGB}{160, 30, 30}
\definecolor{mylightred}{RGB}{255, 150, 150}
\definecolor{myred}{RGB}{200, 110, 110}
\definecolor{myblackred}{RGB}{120, 40, 40}

\definecolor{mypink}{RGB}{255, 30, 80}
\definecolor{myhotpink}{RGB}{255, 80, 200}
\definecolor{mywarmpink}{RGB}{255, 60, 160}
\definecolor{mylightpink}{RGB}{255, 80, 200}
\definecolor{mydarkpink}{RGB}{155, 25, 60}

\definecolor{mydarkcolor}{RGB}{60, 25, 155}
\definecolor{mylightcolor}{RGB}{130, 180, 250}

\setbeamertemplate{itemize items}[default]

\setbeamertemplate{itemize item}{\color{myblackblue}$\blacksquare$}
\setbeamertemplate{itemize subitem}{\color{mydarkblue}$\blacktriangleright$}
\setbeamertemplate{itemize subsubitem}{\color{mygray}$\blacksquare$}

\setbeamercolor{palette quaternary}{fg=white,bg=mygblue} %mydarkgray
\setbeamercolor{titlelike}{parent=palette quaternary}

\setbeamercolor{palette quaternary2}{fg=white,bg=mygblue}%black myblue
\setbeamercolor{frametitle}{parent=palette quaternary2}

\setbeamerfont{frametitle}{size=\Large,series=\scshape}
\setbeamerfont{framesubtitle}{size=\normalsize,series=\upshape}


%%%%%%%%%%%%%%%%%%%%%%%%%%%%
% block settings

%\setbeamercolor{block title}{bg=red!50,fg=black}
%\setbeamercolor{block title}{bg=mylightblue,fg=black}
\setbeamercolor{block title}{bg=myblackblue,fg=white}

\setbeamercolor*{block title example}{bg=mygreen!40!white,fg=black}

\setbeamercolor*{block body example}{fg= black,
	bg= mygreen2}


%%%%%%%%%%%%%%%%%%%%%%%%%%%%
% URL settings
\hypersetup{
	colorlinks=false,
	linkcolor=blue,
	filecolor=blue,      
	urlcolor=blue,
}

%%%%%%%%%%%%%%%%%%%%%%%%%%

\renewcommand{\familydefault}{\rmdefault}

\usepackage{amsmath}
\usepackage{mathtools}

\usepackage{subcaption}

\usepackage{qrcode}

\newcommand{\bo}[1] {\mathbf{#1}}
\newcommand{\R}{\mathbb{R}} 
\newcommand{\T}{^\top}     



\newcommand{\mydate}{Spring 2025}

\newcommand{\mygit}{\textcolor{blue}{\href{https://github.com/SergeiSa/Computational-Intelligence-2025}{github.com/SergeiSa/Computational-Intelligence-2025}}}

\newcommand{\myqr}{ \textcolor{black}{\qrcode[height=1.5in]{https://github.com/SergeiSa/Computational-Intelligence-2025}}
}

\newcommand{\myqrframe}{
	\begin{frame}
		\centerline{Lecture slides are available via Github, links are on Moodle:}
		\bigskip
		\centerline{\mygit}
		\bigskip
		\myqr
	\end{frame}
}


\newcommand{\bref}[2] {\textcolor{blue}{\href{#1}{#2}}}



%%%%%%%%%%%%%%%%%%%%%%%%%%%%
% code settings

\usepackage{listings}
\usepackage{color}
% \definecolor{mygreen}{rgb}{0,0.6,0}
% \definecolor{mygray}{rgb}{0.5,0.5,0.5}
\definecolor{mymauve}{rgb}{0.58,0,0.82}
\lstset{ 
	backgroundcolor=\color{white},   % choose the background color; you must add \usepackage{color} or \usepackage{xcolor}; should come as last argument
	basicstyle=\footnotesize,        % the size of the fonts that are used for the code
	breakatwhitespace=false,         % sets if automatic breaks should only happen at whitespace
	breaklines=true,                 % sets automatic line breaking
	captionpos=b,                    % sets the caption-position to bottom
	commentstyle=\color{mygreen},    % comment style
	deletekeywords={...},            % if you want to delete keywords from the given language
	escapeinside={\%*}{*)},          % if you want to add LaTeX within your code
	extendedchars=true,              % lets you use non-ASCII characters; for 8-bits encodings only, does not work with UTF-8
	firstnumber=0000,                % start line enumeration with line 0000
	frame=single,	                   % adds a frame around the code
	keepspaces=true,                 % keeps spaces in text, useful for keeping indentation of code (possibly needs columns=flexible)
	keywordstyle=\color{blue},       % keyword style
	language=Octave,                 % the language of the code
	morekeywords={*,...},            % if you want to add more keywords to the set
	numbers=left,                    % where to put the line-numbers; possible values are (none, left, right)
	numbersep=5pt,                   % how far the line-numbers are from the code
	numberstyle=\tiny\color{mygray}, % the style that is used for the line-numbers
	rulecolor=\color{black},         % if not set, the frame-color may be changed on line-breaks within not-black text (e.g. comments (green here))
	showspaces=false,                % show spaces everywhere adding particular underscores; it overrides 'showstringspaces'
	showstringspaces=false,          % underline spaces within strings only
	showtabs=false,                  % show tabs within strings adding particular underscores
	stepnumber=2,                    % the step between two line-numbers. If it's 1, each line will be numbered
	stringstyle=\color{mymauve},     % string literal style
	tabsize=2,	                   % sets default tabsize to 2 spaces
	title=\lstname                   % show the filename of files included with \lstinputlisting; also try caption instead of title
}

%%%%%%%%%%%%%%%%%%%%%%%%%%%%
% tikz settings

\usepackage{tikz}
\tikzset{every picture/.style={line width=0.75pt}}

%%%%%%%%%%%%%%%%%%%%%%%%%%%%




\title{SVD}
\subtitle{Computational Intelligence, Lecture 3}
\author{by Sergei Savin}
\centering
\date{\mydate}



\begin{document}
\maketitle





\begin{frame}{Singular Value Decomposition}
	% \framesubtitle{Local coordinates}
	\begin{flushleft}
		
		Given $\bo{A} \in \R^{n, m}$ we can find its Singular Value Decomposition (SVD):
		
		\begin{equation}
			\bo{A} = 
			\begin{bmatrix}
				\bo{C} & \bo{L}
			\end{bmatrix}
			\begin{bmatrix}
				\bo{\Sigma} & \bo{0} \\
				\bo{0} & \bo{0}
			\end{bmatrix}
			\begin{bmatrix}
				\bo{R}^\top \\ \bo{N}^\top
			\end{bmatrix}
		\end{equation}
		
		\begin{equation}
			\bo{A} = 
			\bo{C} \bo{\Sigma} \bo{R}^\top
		\end{equation}
		
		where $\bo{C}$, $\bo{L}$, $\bo{R}$ and $\bo{N}$ are column, left null, row and null space bases (orthonormal), $\bo{\Sigma}$ is the diagonal matrix of singular values. The singular values are positive and are sorted in the decreasing order.
		
		\bigskip

		Rank of the matrix is computed as the size of $\bo{\Sigma}$. Note that numeric tolerance applies when deciding if the singular value is non-zero.
		
	\end{flushleft}
\end{frame}




\begin{frame}{SVD of a transpose}
	% \framesubtitle{Local coordinates}
	\begin{flushleft}
		
		Let us find SVD decomposition of a $\bo{A}^\top$:
		
		\begin{equation}
			\bo{A}^\top = 
			\begin{bmatrix}
				\bo{C}_t & \bo{L}_t
			\end{bmatrix}
			\begin{bmatrix}
				\bo{\Sigma}_t & \bo{0} \\
				\bo{0} & \bo{0}
			\end{bmatrix}
			\begin{bmatrix}
				\bo{R}_t^\top \\ \bo{N}_t^\top
			\end{bmatrix}
		\end{equation}
		
		Let us transpose it (remembering that transpose of a diagonal matrix the original matrix $\bo{\Sigma}_t^\top = \bo{\Sigma}_t$):
		
		
		\begin{equation}
			\bo{A} = 
			\begin{bmatrix}
				\bo{R}_t & \bo{N}_t
			\end{bmatrix}
			\begin{bmatrix}
				\bo{\Sigma}_t & \bo{0} \\
				\bo{0} & \bo{0}
			\end{bmatrix}
			\begin{bmatrix}
				\bo{C}_t^\top \\ \bo{L}_t^\top
			\end{bmatrix}
		\end{equation}
		
		Thus we can see that the row space of the original matrix $\bo{A}$ is the column space of the transpose $\bo{A}^\top$. And the left null space of the original matrix $\bo{A}$ is the null space of the transpose $\bo{A}^\top$.
		
		
	\end{flushleft}
\end{frame}



\begin{frame}{Pseudoinverse via SVD, 1}
	% \framesubtitle{Local coordinates}
	\begin{flushleft}
		
		Let us compute least squares - minimum of $e = ||\bo{A}\bo{x} - \bo{y}||_2$. We find extremum:
		%
		\begin{align}
			\frac{d}{d \bo{x}} \left( (\bo{A}\bo{x} - \bo{y})^\top(\bo{A}\bo{x} - \bo{y}) \right) 
			= 0
			\\
			2\bo{A}^\top(\bo{A}\bo{x} - \bo{y}) = 0
			\\
			\bo{A}^\top\bo{A}\bo{x} = \bo{A}^\top\bo{y}
		\end{align}
		
		We find SVD decomposition of $\bo{A} = \bo{C} \bo{\Sigma} \bo{R}^\top$:
		%
		\begin{align}
			\bo{R}  \bo{\Sigma} \bo{C}\T   \bo{C} \bo{\Sigma} \bo{R}^\top \bo{x} = \bo{R}  \bo{\Sigma} \bo{C}\T  \bo{y}
			\\
			\bo{R}  \bo{\Sigma} \bo{\Sigma} \bo{R}^\top \bo{x} = \bo{R}  \bo{\Sigma} \bo{C}\T  \bo{y}
		\end{align}
		
	\end{flushleft}
\end{frame}





\begin{frame}{Pseudoinverse via SVD, 2}
	% \framesubtitle{Local coordinates}
	\begin{flushleft}
		
		Since both sides lie in the column space of $\bo{R}$, we can multiply by $\bo{R}\T$:
		%
		\begin{align}
			\bo{R}\T\bo{R}  \bo{\Sigma} \bo{\Sigma} \bo{R}^\top \bo{x} = \bo{R}\T\bo{R}  \bo{\Sigma} \bo{C}\T  \bo{y}
			\\
			\bo{\Sigma} \bo{\Sigma} \bo{R}^\top \bo{x} = \bo{\Sigma} \bo{C}\T  \bo{y}
			\\
			\bo{R}^\top \bo{x} = \bo{\Sigma}^{-1} \bo{C}\T  \bo{y}
		\end{align}
		
		Since $\bo{R}$ and $\bo{N}$ are orthogonal compliments, we can represent $\bo{x}$ as its decomposition: $\bo{x} = \bo{N}\bo{z} + \bo{R}\zeta$:
		%
		\begin{align}
			\bo{R}^\top \bo{N}\bo{z} + \bo{R}^\top \bo{R}\zeta = \bo{\Sigma}^{-1} \bo{C}\T  \bo{y}
			\\
			\zeta = \bo{\Sigma}^{-1} \bo{C}\T  \bo{y}
		\end{align}
	
		With that we can compute $\bo{x}$:
		%
		\begin{align}
			\bo{x} = \bo{N}\bo{z} + \bo{R}\bo{\Sigma}^{-1} \bo{C}\T  \bo{y}
		\end{align}
		
	\end{flushleft}
\end{frame}



\begin{frame}{Pseudoinverse via SVD, 3}
	% \framesubtitle{Local coordinates}
	\begin{flushleft}
		
		Expression $\bo{x} = \bo{N}\bo{z} + \bo{R}\bo{\Sigma}^{-1} \bo{C}\T  \bo{y}$ gives us all least-residual solutions.
		
		\bigskip
		
		Since $\bo{N}\bo{z}$ is orthogonal to $\bo{R}\bo{\Sigma}^{-1} \bo{C}\T  \bo{y}$, we conclude that least-norm solution is given as:
		%
		\begin{align}
			\bo{x} = \bo{R}\bo{\Sigma}^{-1} \bo{C}\T  \bo{y}
		\end{align}
		
		With that we can define pseudoinverse matrix $\bo{A}^+$ as:
		
		\begin{equation}
			\bo{A}^+ = 
			\bo{R} \bo{\Sigma}^{-1} \bo{C}^\top
		\end{equation}		
		
		Note that this proves that $\bo{A}^+$ lies in the row space of $\bo{A}$.
		
	\end{flushleft}
\end{frame}








\begin{frame}{Projectors (1)}
	% \framesubtitle{Local coordinates}
	\begin{flushleft}
		
		Let us prove that $\bo{A}\bo{A}^+$ is equivalent to $\bo{C}\bo{C}^\top$:
		
		\begin{equation}
			\bo{A} \bo{A}^+ = 
			\bo{C} \bo{\Sigma} \bo{R}^\top 
			\bo{R} \bo{\Sigma}^{-1} \bo{C}^\top
		\end{equation}
		
		
		\begin{equation}
			\bo{A} \bo{A}^+ = 
			\bo{C} \bo{\Sigma} \bo{\Sigma}^{-1} \bo{C}^\top
		\end{equation}
		
		\begin{equation}
			\bo{A} \bo{A}^+ = 
			\bo{C} \bo{C}^\top
		\end{equation}
		
		
	\end{flushleft}
\end{frame}



\begin{frame}{Projectors (2)}
	% \framesubtitle{Local coordinates}
	\begin{flushleft}
		
		Let us prove that $\bo{A}^+\bo{A}$ is equivalent to $\bo{R}\bo{R}^\top$:
		
		\begin{equation}
			\bo{A}^+ \bo{A} = 
			\bo{R} \bo{\Sigma}^{-1} \bo{C}^\top
			\bo{C} \bo{\Sigma} \bo{R}^\top 
		\end{equation}
		
		\begin{equation}
			\bo{A}^+ \bo{A} = 
			\bo{R} \bo{\Sigma}^{-1} \bo{\Sigma} \bo{R}^\top 
		\end{equation}		
		
		\begin{equation}
			\bo{A}^+ \bo{A} = 
			\bo{R} \bo{R}^\top 
		\end{equation}			
		
	\end{flushleft}
\end{frame}





\begin{frame}{Projectors (3)}
	% \framesubtitle{Local coordinates}
	\begin{flushleft}
		
		Let us denote $\bo{P} = \bo{A}\bo{A}^+$. 
		Let's prove that $\bo{P}\bo{P} = \bo{P}$:
		
		\begin{align}
			\bo{A} \bo{A}^+ \bo{A} \bo{A}^+ &= 
			\bo{C} \bo{\Sigma} \bo{R}^\top 
			\bo{R} \bo{\Sigma}^{-1} \bo{C}^\top 
			\bo{C} \bo{\Sigma} \bo{R}^\top 
			\bo{R} \bo{\Sigma}^{-1} \bo{C}^\top
			\\
			%
			\bo{A} \bo{A}^+ \bo{A} \bo{A}^+ &= 
			\bo{C} \bo{\Sigma} \bo{\Sigma}^{-1}\bo{\Sigma} \bo{\Sigma}^{-1} \bo{C}^\top
			\\
			%
			\bo{A} \bo{A}^+ \bo{A} \bo{A}^+ &= 
			\bo{C} \bo{C}^\top
			\\
			%
			\bo{A} \bo{A}^+ \bo{A} \bo{A}^+ &= 
			\bo{A} \bo{A}^+
		\end{align}
		
		
		The same is true for $\bo{P} = \bo{A}^+\bo{A}$: we can prove that $\bo{A}^+\bo{A} \bo{A}^+\bo{A} = \bo{A}^+\bo{A}$.
		
	\end{flushleft}
\end{frame}



\begin{frame}{Transpose of a projector}
	% \framesubtitle{Local coordinates}
	\begin{flushleft}
		
		Let's prove that $\bo{P}\T = \bo{P}$. 
		
		Again, we use the fact that $\bo{P} = \bo{C}\bo{C}\T$.
		
		\begin{align}
			\bo{P}\T = (\bo{C}\bo{C}\T)\T=\bo{C}\bo{C}\T=\bo{P}. \qed
		\end{align}
		
		
	\end{flushleft}
\end{frame}



\begin{frame}{Pseudoinverse of a projector}
	\begin{flushleft}
		
		Let's prove that $\bo{P}^+ = \bo{P}$. 
		
		\bigskip
		
		First, we find a basis in the linear space where $\bo{P}$ projects onto: $\bo{C} = \text{col}(\bo{P})$, therefore $\bo{P} = \bo{C}\bo{C}\T = \bo{C}\bo{I}\bo{C}\T$, which is an SVD decomposition of $\bo{P}$. But we know how to find a pseudoinverse of a linear operator, given its SVD decomposition:
		%
		\begin{align}
			\bo{P} &= \bo{C}\bo{I}\bo{C}\T,
			\\
			\bo{P}^+ &= \bo{C}\bo{I}^{-1}\bo{C}\T,
			\\
			\bo{P}^+ &= \bo{C}\bo{C}\T,
			\\
			\bo{P}^+ &= \bo{P}. \ \qed
		\end{align}
		
		
	\end{flushleft}
\end{frame}




\begin{frame}{Example 1, end-effector}
	\begin{flushleft}
		
		Consider a 6 DOF manipulator described by joint coordinates $\bo{q}$,  with a end-effector whose position is given by vector $\bo{r} = \bo{r}(\bo{q})$. Find all movements that do not displace the end-effector, which are available to us at a configuration $\bo{q}_0$.
		
		\bigskip
		
		\emph{Solution.} We can find velocity of the end-effector 
		
		\begin{equation}
			\bo{v} = \frac{d}{dt} \bo{r}(\bo{q}) = \frac{\partial \bo{r}}{\partial \bo{q}} \frac{\partial \bo{q}}{\partial t} = \bo{J} \dot{\bo{q}}
		\end{equation}
		
		"Movements that do not displace the end-effector" describes joint velocities that do not produce end-effector velocities - in other words, all joint velocities in the null space of $\bo{J}$ :
		
		\begin{align}
			\bo{N} &= \text{null} ( \bo{J} )
			\\
			 \dot{\bo{q}} &= \bo{N} \bo{z}, \ \forall \bo{z} \ \ \ \ \ \textcolor{mygray}{(answer)}
		\end{align}
		
		
		
	\end{flushleft}
\end{frame}



\begin{frame}{Example 2, singularity}
	\begin{flushleft}
		
		Consider a 6 DOF manipulator described by joint coordinates $\bo{q}$,  with a end-effector whose position is given by the vector $\bo{r} = \bo{r}(\bo{q})$. Check if the configuration $\bo{q}_0$ puts the manipulator in such a singular position that the end effector lost one of its positional degrees of freedom.
		
		\bigskip
		
		\emph{Solution.} As before, $\bo{v} = \bo{J} \dot{\bo{q}}$.
		
		"the end effector lost one of its positional degrees of freedom" means that the space of possible velocities of the end-effector has dimensionality 2 or less. But since $\bo{v} = \bo{J} \dot{\bo{q}}$, it is enough to check the space of all posssible outputs of $\bo{J}$ :
		
		\begin{align}
			 \text{orth} ( \bo{J} ) = 	\bo{C} \in \R^{3, k}
			\\
			\text{if} \ \ k < 3 \ \ \text{then the answer is YES, else NO}
		\end{align}
		
		
		
	\end{flushleft}
\end{frame}




\begin{frame}{Example 3, drop in DOF}
	\begin{flushleft}
		
		Consider a 7 DOF manipulator described by joint coordinates $\bo{q}$, with an end-effector whose velocity is given by the expression $\bo{v} = \bo{J} \dot{\bo{q}}$ and whose angular velocity is given as $\bo{w} = \bo{J}_w \dot{\bo{q}}$. If we require that $\textcolor{myred}{\dot q_2  = \dot q_3}$, does the end-effector still have 6 DOF in the configuration $\bo{q}_0$?
		
		\bigskip
		
		\emph{Solution.} Consider new independent generalized velocities $\dot{\bo{g}}$ defined as:
		
		\begin{align}
			\dot{\bo{q}}
			=
			\begin{bmatrix}
				\dot q_1 \\ 	\textcolor{myred}{\dot q_2} \\ 	\textcolor{myred}{\dot q_3} \\ 	\dot q_4 \\ 	\dot q_5 \\ 	\dot q_6 \\ 	\dot q_7 
			\end{bmatrix}
			= 
			\begin{bmatrix}
			\dot g_1 \\ 
			\textcolor{myred}{\dot g_2}	 \\ \textcolor{myred}{\dot g_2}	 \\ 	\dot g_3 \\ 		\dot g_4 \\ 	\dot g_5\\ 	\dot g_6
			\end{bmatrix}
			=
			\begin{bmatrix}
				1 & 0& 0& 0& 0& 0 \\
				0 & 1& 0& 0& 0& 0 \\
				0 & 1& 0& 0& 0& 0 \\
				0 & 0& 1& 0& 0& 0 \\
				0 & 0& 0& 1& 0& 0 \\
				0 & 0& 0& 0& 1& 0 \\
				0 & 0& 0& 0& 0& 1 \\
			\end{bmatrix}
			\begin{bmatrix}
				\dot g_1 \\ 	\dot g_2 \\ 	\dot g_3 \\ 		\dot g_4 \\ 	\dot g_5\\ 	\dot g_6
			\end{bmatrix} 
			= 
			\bo{M}\dot{\bo{g}}
		\end{align}
		
		
		
		
	\end{flushleft}
\end{frame}



\begin{frame}{Example 3, drop in DOF}
	\begin{flushleft}
		
		Since $\dot{\bo{q}} = \bo{M}\dot{\bo{g}}$ we can write:
		
		\begin{align}
			\bo{v} = \bo{J} \dot{\bo{q}} = \bo{J} \bo{M}\dot{\bo{g}}
			\\
			\bo{w} = \bo{J}_w \dot{\bo{q}}= \bo{J}_w \bo{M}\dot{\bo{g}}
		\end{align}
		
		In the matrix form:
		
		\begin{align}
			\begin{bmatrix}
					\bo{v} \\ \bo{w}
			\end{bmatrix}
			=
			\begin{bmatrix}
				\bo{J} \bo{M} \\ \bo{J}_w \bo{M}
			\end{bmatrix}
		\dot{\bo{g}}
		\end{align}
		
		\begin{align}
					\text{if} \ \ \ 
					\text{dim}   \left(  \text{orth}\left( \begin{bmatrix}
				\bo{J} \bo{M} \\ \bo{J}_w \bo{M}
			\end{bmatrix} \right) \right)
			= 6 \ \ \ \text{then the answer is YES}
		\end{align}
		

		
	\end{flushleft}
\end{frame}




\begin{frame}{Inverse Dynamics, 1}
	%\framesubtitle{How do we know the state?}
	\begin{flushleft}
		
		Consider a dynamical system:
		%
		\begin{equation}
			\bo{H} \ddot{\bo{q}} + \bo{C}\dot{\bo{q}} + \bo{g} = \bo{B}\bo{u}
		\end{equation}		
		
		\textbf{Task 1}: 
		For the current state $\bo{q}$, $\dot{\bo{q}}$ find such $\bo{u}$ (if it exists) that $\ddot{\bo{q}} = \bo{a}$.
		
		\bigskip
		
		\textbf{Solution}. The solution will exist if the vector $\bo{r} = \bo{H} \bo{a} + \bo{C}\dot{\bo{q}} + \bo{g} $ lies in the column space on $\bo{B}$. The condition for the vector to be in the column space of $\bo{B}$ is:
		%
		\begin{equation}
			(\bo{I} - \bo{B}\bo{B}^+)\bo{r} = 0
		\end{equation}		
		%
		If the solution exists, the it takes form:
		%
		\begin{equation}
			\bo{u} = \bo{B}^+ (\bo{H} \bo{a} + \bo{C}\dot{\bo{q}} + \bo{g} )
		\end{equation}		
		
	\end{flushleft}
\end{frame}




\begin{frame}{Inverse Dynamics, 2}
	%\framesubtitle{How do we know the state?}
	\begin{flushleft}
		
		\textbf{Task 2}: 
		For the current state $\bo{q}$, $\dot{\bo{q}}$ find if there exist multiple controls $\bo{u}$ that make $\ddot{\bo{q}} = \bo{a}$. If yes, find all of them.
		
		\bigskip
		
		\textbf{Solution}. If $(\bo{I} - \bo{B}\bo{B}^+)\bo{r} = 0$ then at least one solution to the ID exists. Multiple solutions will exist iff $\bo{B}$ has a non-trivial null space:
		%
		\begin{equation}
			\text{dim}(\text{null}(\bo{B})) = k > 0
		\end{equation}		
		%
		If the null space of $\bo{B}$ is non-trivial, then all solutions to the ID take form:
		%
		\begin{align}
			\bo{u} = \bo{B}^+ (\bo{H} \bo{a} + \bo{C}\dot{\bo{q}} + \bo{g} ) + 
			\bo{N}\bo{v}, \ \ \ \forall \bo{v} \in \R^k
			\\
			\bo{N} = \text{null}(\bo{B})
		\end{align}		
		
	\end{flushleft}
\end{frame}




\begin{frame}{Further reading}
	% \framesubtitle{Local coordinates}
	\begin{flushleft}
		
		\begin{itemize}
			\item \bref{https://faculty.math.illinois.edu/~mlavrov/docs/484-spring-2019/ch4lec4.pdf}{Minimum Norm Solutions, Math 484: Nonlinear Programming, Mikhail Lavrov}
			%https://faculty.math.illinois.edu/~mlavrov/courses/484-spring-2019.html
			
			\item \bref{https://faculty.math.illinois.edu/~mlavrov/docs/484-spring-2019/ch4lec3.pdf}{Orthogonality, Math 484: Nonlinear Programming, Mikhail Lavrov}
			
			\item \bref{http://databookuw.com/databook.pdf}{Data Driven Science \& Engineering. Machine Learning, Dynamical Systems, and Control, Steven L. Brunton, J. Nathan Kutz}, chapter Singular Value Decomposition (SVD)
			
		\end{itemize}
		
		
	\end{flushleft}
\end{frame}







\myqrframe




\end{document}
