\documentclass{beamer}

\pdfmapfile{+sansmathaccent.map}


\mode<presentation>
{
	\usetheme{Warsaw} % or try Darmstadt, Madrid, Warsaw, Rochester, CambridgeUS, ...
	\usecolortheme{seahorse} % or try seahorse, beaver, crane, wolverine, ...
	\usefonttheme{serif}  % or try serif, structurebold, ...
	\setbeamertemplate{navigation symbols}{}
	\setbeamertemplate{caption}[numbered]
} 


%%%%%%%%%%%%%%%%%%%%%%%%%%%%
% itemize settings

\definecolor{mypaleblue}{RGB}{240, 240, 255}
\definecolor{mylightblue}{RGB}{120, 150, 255}
\definecolor{myblue}{RGB}{90, 90, 255}
\definecolor{mygblue}{RGB}{70, 110, 240}
\definecolor{mydarkblue}{RGB}{0, 0, 180}
\definecolor{myblackblue}{RGB}{40, 40, 120}

\definecolor{mygreen}{RGB}{0, 200, 0}
\definecolor{mydarkgreen}{RGB}{0, 120, 0}
\definecolor{mygreen2}{RGB}{245, 255, 230}

\definecolor{mygray}{gray}{0.8}
\definecolor{mygray2}{RGB}{130, 130, 130}
\definecolor{mydarkgray}{RGB}{80, 80, 160}
\definecolor{mylightgray}{RGB}{160, 160, 160}

\definecolor{mydarkred}{RGB}{160, 30, 30}
\definecolor{mylightred}{RGB}{255, 150, 150}
\definecolor{myred}{RGB}{200, 110, 110}
\definecolor{myblackred}{RGB}{120, 40, 40}

\definecolor{mypink}{RGB}{255, 30, 80}
\definecolor{myhotpink}{RGB}{255, 80, 200}
\definecolor{mywarmpink}{RGB}{255, 60, 160}
\definecolor{mylightpink}{RGB}{255, 80, 200}
\definecolor{mydarkpink}{RGB}{155, 25, 60}

\definecolor{mydarkcolor}{RGB}{60, 25, 155}
\definecolor{mylightcolor}{RGB}{130, 180, 250}

\setbeamertemplate{itemize items}[default]

\setbeamertemplate{itemize item}{\color{myblackblue}$\blacksquare$}
\setbeamertemplate{itemize subitem}{\color{mydarkblue}$\blacktriangleright$}
\setbeamertemplate{itemize subsubitem}{\color{mygray}$\blacksquare$}

\setbeamercolor{palette quaternary}{fg=white,bg=mygblue} %mydarkgray
\setbeamercolor{titlelike}{parent=palette quaternary}

\setbeamercolor{palette quaternary2}{fg=white,bg=mygblue}%black myblue
\setbeamercolor{frametitle}{parent=palette quaternary2}

\setbeamerfont{frametitle}{size=\Large,series=\scshape}
\setbeamerfont{framesubtitle}{size=\normalsize,series=\upshape}


%%%%%%%%%%%%%%%%%%%%%%%%%%%%
% block settings

%\setbeamercolor{block title}{bg=red!50,fg=black}
%\setbeamercolor{block title}{bg=mylightblue,fg=black}
\setbeamercolor{block title}{bg=myblackblue,fg=white}

\setbeamercolor*{block title example}{bg=mygreen!40!white,fg=black}

\setbeamercolor*{block body example}{fg= black,
	bg= mygreen2}


%%%%%%%%%%%%%%%%%%%%%%%%%%%%
% URL settings
\hypersetup{
	colorlinks=false,
	linkcolor=blue,
	filecolor=blue,      
	urlcolor=blue,
}

%%%%%%%%%%%%%%%%%%%%%%%%%%

\renewcommand{\familydefault}{\rmdefault}

\usepackage{amsmath}
\usepackage{mathtools}

\usepackage{subcaption}

\usepackage{qrcode}

\newcommand{\bo}[1] {\mathbf{#1}}
\newcommand{\R}{\mathbb{R}} 
\newcommand{\T}{^\top}     



\newcommand{\mydate}{Spring 2025}

\newcommand{\mygit}{\textcolor{blue}{\href{https://github.com/SergeiSa/Computational-Intelligence-2025}{github.com/SergeiSa/Computational-Intelligence-2025}}}

\newcommand{\myqr}{ \textcolor{black}{\qrcode[height=1.5in]{https://github.com/SergeiSa/Computational-Intelligence-2025}}
}

\newcommand{\myqrframe}{
	\begin{frame}
		\centerline{Lecture slides are available via Github, links are on Moodle:}
		\bigskip
		\centerline{\mygit}
		\bigskip
		\myqr
	\end{frame}
}


\newcommand{\bref}[2] {\textcolor{blue}{\href{#1}{#2}}}



%%%%%%%%%%%%%%%%%%%%%%%%%%%%
% code settings

\usepackage{listings}
\usepackage{color}
% \definecolor{mygreen}{rgb}{0,0.6,0}
% \definecolor{mygray}{rgb}{0.5,0.5,0.5}
\definecolor{mymauve}{rgb}{0.58,0,0.82}
\lstset{ 
	backgroundcolor=\color{white},   % choose the background color; you must add \usepackage{color} or \usepackage{xcolor}; should come as last argument
	basicstyle=\footnotesize,        % the size of the fonts that are used for the code
	breakatwhitespace=false,         % sets if automatic breaks should only happen at whitespace
	breaklines=true,                 % sets automatic line breaking
	captionpos=b,                    % sets the caption-position to bottom
	commentstyle=\color{mygreen},    % comment style
	deletekeywords={...},            % if you want to delete keywords from the given language
	escapeinside={\%*}{*)},          % if you want to add LaTeX within your code
	extendedchars=true,              % lets you use non-ASCII characters; for 8-bits encodings only, does not work with UTF-8
	firstnumber=0000,                % start line enumeration with line 0000
	frame=single,	                   % adds a frame around the code
	keepspaces=true,                 % keeps spaces in text, useful for keeping indentation of code (possibly needs columns=flexible)
	keywordstyle=\color{blue},       % keyword style
	language=Octave,                 % the language of the code
	morekeywords={*,...},            % if you want to add more keywords to the set
	numbers=left,                    % where to put the line-numbers; possible values are (none, left, right)
	numbersep=5pt,                   % how far the line-numbers are from the code
	numberstyle=\tiny\color{mygray}, % the style that is used for the line-numbers
	rulecolor=\color{black},         % if not set, the frame-color may be changed on line-breaks within not-black text (e.g. comments (green here))
	showspaces=false,                % show spaces everywhere adding particular underscores; it overrides 'showstringspaces'
	showstringspaces=false,          % underline spaces within strings only
	showtabs=false,                  % show tabs within strings adding particular underscores
	stepnumber=2,                    % the step between two line-numbers. If it's 1, each line will be numbered
	stringstyle=\color{mymauve},     % string literal style
	tabsize=2,	                   % sets default tabsize to 2 spaces
	title=\lstname                   % show the filename of files included with \lstinputlisting; also try caption instead of title
}

%%%%%%%%%%%%%%%%%%%%%%%%%%%%
% tikz settings

\usepackage{tikz}
\tikzset{every picture/.style={line width=0.75pt}}

%%%%%%%%%%%%%%%%%%%%%%%%%%%%




\title{Examples}
\subtitle{Computational Intelligence, lecture ??}
\author{by Sergei Savin}
\centering
\date{\mydate}



\begin{document}
\maketitle


\begin{frame}{Content}

\begin{itemize}
\item  Converting State-Space to ODE
\end{itemize}

\end{frame}




\begin{frame}{Inverse Dynamics, 1}
	%\framesubtitle{How do we know the state?}
	\begin{flushleft}
		
		Consider a dynamical system:
		%
		\begin{equation}
			\bo{H} \ddot{\bo{q}} + \bo{C}\dot{\bo{q}} + \bo{g} = \bo{B}\bo{u}
		\end{equation}		
		
		\textbf{Task 1}: 
		For the current state $\bo{q}$, $\dot{\bo{q}}$ find such $\bo{u}$ (if it exists) that $\ddot{\bo{q}} = \bo{a}$.
		
		\bigskip
		
		\textbf{Solution}. The solution will exist if the vector $\bo{r} = \bo{H} \bo{a} + \bo{C}\dot{\bo{q}} + \bo{g} $ lies in the column space on $\bo{B}$. The condition for the vector to be in the column space of $\bo{B}$ is:
		%
		\begin{equation}
			(\bo{I} - \bo{B}\bo{B}^+)\bo{r} = 0
		\end{equation}		
		%
		If the solution exists, the it takes form:
		%
		\begin{equation}
			\bo{u} = \bo{B}^+ (\bo{H} \bo{a} + \bo{C}\dot{\bo{q}} + \bo{g} )
		\end{equation}		
		
	\end{flushleft}
\end{frame}




\begin{frame}{Inverse Dynamics, 2}
	%\framesubtitle{How do we know the state?}
	\begin{flushleft}
		
		\textbf{Task 2}: 
		For the current state $\bo{q}$, $\dot{\bo{q}}$ find if there exist multiple controls $\bo{u}$ that make $\ddot{\bo{q}} = \bo{a}$. If yes, find all of them.
		
		\bigskip
		
		\textbf{Solution}. If $(\bo{I} - \bo{B}\bo{B}^+)\bo{r} = 0$ then at least one solution to the ID exists. Multiple solutions will exist iff $\bo{B}$ has a non-trivial null space:
		%
		\begin{equation}
			\text{dim}(\text{null}(\bo{B})) = k > 0
		\end{equation}		
		%
		If the null space of $\bo{B}$ is non-trivial, then all solutions to the ID take form:
		%
		\begin{align}
			\bo{u} = \bo{B}^+ (\bo{H} \bo{a} + \bo{C}\dot{\bo{q}} + \bo{g} ) + 
			\bo{N}\bo{v}, \ \ \ \forall \bo{v} \in \R^k
			\\
			\bo{N} = \text{null}(\bo{B})
		\end{align}		
		
	\end{flushleft}
\end{frame}




\begin{frame}{Priority control input, 1}
	%\framesubtitle{How do we know the state?}
	\begin{flushleft}
		
		
		Consider a system with two inputs $\bo{u}_1$ and $\bo{u}_2$:
		%
		\begin{equation}
			\bo{H} \ddot{\bo{q}} + \bo{C}\dot{\bo{q}} + \bo{g} = \bo{B}_1\bo{u}_1 + \bo{B}_2\bo{u}_2
		\end{equation}		
		
		We would like to solve inverse dynamics without using $\bo{u}_2$ unless we have to.
		
		\textbf{Task 1}:  Solve inverse dynamics - for the current state $\bo{q}$, $\dot{\bo{q}}$ find such $\bo{u}_1$ and $\bo{u}_2$ (if it exists) that $\ddot{\bo{q}} = \bo{a}$.
		
		First we check if such pair $(\bo{u}_1, \ \bo{u}_2)$ that $\ddot{\bo{q}} = \bo{a}$ exists. We re-write the dynamics as:
		%
		\begin{equation}
			\bo{r} = 
			\begin{bmatrix}
				\bo{B}_1 & \bo{B}_2
			\end{bmatrix}
			\begin{bmatrix}
			\bo{u}_1 \\ \bo{u}_2
			\end{bmatrix}
		\end{equation}		
		
		with that, we find the criterion for the existence of the ID solution:
		%
		\begin{equation}
			\left
			(\bo{I} - \begin{bmatrix}
				\bo{B}_1 & \bo{B}_2
			\end{bmatrix}
			\begin{bmatrix}
				\bo{B}_1 & \bo{B}_2
			\end{bmatrix}^+
			\right) \bo{r} 
		= 0
		\end{equation}		
		
		
	\end{flushleft}
\end{frame}



\begin{frame}{Priority control input, 2}
	%\framesubtitle{How do we know the state?}
	\begin{flushleft}
		
		We solve the following optimization problem:
		%
		\begin{equation}
			\begin{aligned}
				& \underset{\bo{u}_1, \bo{u}_2}{\text{minimize}}
				& & \frac{1}{2} \bo{u}_2\T \bo{u}_2, \\
				& \text{subject to}
				& & \bo{r} = \bo{B}_1\bo{u}_1 + \bo{B}_2\bo{u}_2
			\end{aligned}
		\end{equation}
		
		We find Lagrangian $L = \frac{1}{2} \bo{u}_2\T \bo{u}_2 + \lambda\T (\bo{B}_1\bo{u}_1 + \bo{B}_2\bo{u}_2 - \bo{r})$.
		
		Partial derivatives of $L$ should be zero at an extremum point:
		%
		\begin{align}
			\frac{\partial L}{\partial \bo{u}_1} &=
			\lambda\T \bo{B}_1 = 0
			\\
			\frac{\partial L}{\partial \bo{u}_2} &=
			\bo{u}_2\T +
			\lambda\T \bo{B}_2 = 0
			\\
			\frac{\partial L}{\partial \lambda} &=
			\bo{B}_1\bo{u}_1 + \bo{B}_2\bo{u}_2 - \bo{r} = 0
		\end{align}
		
		
	\end{flushleft}
\end{frame}



\begin{frame}{Priority control input, 3}
	%\framesubtitle{How do we know the state?}
	\begin{flushleft}
		
		This is the same as:
		%
		\begin{equation}
			\begin{bmatrix}
				\bo{B}_1 & \bo{B}_2 & 0 \\
				0 & 0 & \bo{B}_1\T \\
				0 & \bo{I} & \bo{B}_2\T
			\end{bmatrix}
			\begin{bmatrix}
			\bo{u}_1  \\
			\bo{u}_2  \\
			\lambda 
			\end{bmatrix}
			=
			\begin{bmatrix}
			\bo{r} \\
			0  \\
			0 
			\end{bmatrix}
		\end{equation}
		
		We solve this linear system to find the desired control.
		
		
	\end{flushleft}
\end{frame}







\begin{frame}{State Space}
	%\framesubtitle{How do we know the state?}
	\begin{flushleft}
		
		Consider a state-space representation of a dynamical system:
		
		\begin{equation}
			\begin{cases}
				\dot{\bo{x}} = \bo{A}\bo{x}+\bo{B}\bo{u} 
				\\
				\bo{y} = \bo{C}\bo{x}
			\end{cases}
		\end{equation}
		
		We can introduce a change of variables $\bo{x} = \bo{T}\bo{z}$ where $\text{det}(\bo{T}) \neq 0$:
		
		\begin{equation}
			\begin{cases}
				\dot{\bo{z}} = \bo{T}^{-1}\bo{A}\bo{T} \bo{z}+\bo{T}^{-1}\bo{B}\bo{u} 
				\\
				\bo{y} = \bo{C}\bo{T}\bo{z}
			\end{cases}
		\end{equation}
		
		The two systems are equivalent, with new matrices $\bar{\bo{A}} = \bo{T}^{-1}\bo{A}\bo{T}$, $\bar{\bo{B}} = \bo{T}^{-1}\bo{B}$ and $\bar{\bo{C}} = \bo{C}\bo{T}$.
		
	\end{flushleft}
\end{frame}


\begin{frame}{State Space to ODE}
	%\framesubtitle{How do we know the state?}
	\begin{flushleft}
		
		Given a state-space representation of the form:
		%
		\begin{equation}
			\begin{cases}
				\begin{bmatrix}
					\dot x_1 \\ \dot x_2 \\  ... \\ \dot x_{n-1} \\ \dot x_n
				\end{bmatrix}
				\begin{bmatrix}
					0 & 1 & ... & 0 & 0 \\
					0 & 0& ... & 0 & 0 \\
					... & ... & ... & ... & ...  \\
					0 & 0& ... & 0 & 1 \\
					a_1 & a_2 & ... & a_{n-1} & a_n 
				\end{bmatrix}
				\begin{bmatrix}
					x_1 \\ x_2 \\  ... \\ x_{n-1} \\  x_n
				\end{bmatrix}
				+
				\begin{bmatrix}
					0 \\ 0 \\  ... \\ 0 \\  b_n
				\end{bmatrix}
				u
				\\
				y = 
				\begin{bmatrix}
					1 &0 &  ... &0 &  0
				\end{bmatrix}
				\begin{bmatrix}
				x_1 \\ x_2 \\  ... \\ x_{n-1} \\  x_n
				\end{bmatrix}
			\end{cases}
		\end{equation}
		
		We can re-write it in an ODE form:
		%
		\begin{equation}
			y^{(n)} = a_1 y + a_2 \dot y + ... + a_n y^{(n-1)} + b_n u
		\end{equation}
		
		
	\end{flushleft}
\end{frame}


\begin{frame}{State Space to ODE, 1}
	%\framesubtitle{How do we know the state?}
	\begin{flushleft}
		
		As long as state-space system has matrices:
		%
		\begin{align}
			\bar{\bo{A}} = 
			\begin{bmatrix}
				\bo{0}_{(n-1) \times 1} & \bo{I}_{(n-1) \times (n-1)} \\
				\alpha & \beta\T
			\end{bmatrix},
			\ \ \
			\bar{\bo{B}} = 
			\begin{bmatrix}
				\bo{0}_{(n-1) \times 1} \\
				1
			\end{bmatrix}
		\end{align}
		%
		we can re-write it in an ODE form. 
		
		\bigskip
		
		We can formulate a task: given matrices $\bo{A}$ and $\bo{B}$ find such transformation $\bo{T} = 
		\begin{bmatrix}
			\bo{t}_1 & ... & \bo{t}_n
		\end{bmatrix}$ 
		that $\bar{\bo{A}} = \bo{T}^{-1}\bo{A}\bo{T}$, $\bar{\bo{B}} = \bo{T}^{-1}\bo{B}$.
		
		
	\end{flushleft}
\end{frame}



\begin{frame}{State Space to ODE, 2}
	%\framesubtitle{How do we know the state?}
	\begin{flushleft}
		
		Consider the equality $\bar{\bo{B}} = \bo{T}^{-1}\bo{B}$. Since $ \bo{T}$ is full rank, we can re-write the equality:
		%
		\begin{align}
			\bo{T}\bar{\bo{B}} = \bo{B}
			\\
			\bo{t}_n = \bo{B}
		\end{align}
		
		Let us define $\bo{P} = 
		\begin{bmatrix}
			\bo{t}_1 & ... & \bo{t}_{n-1}
		\end{bmatrix}$, thus giving us $\bo{T} = 
		\begin{bmatrix}
		\bo{P}& \bo{B}
		\end{bmatrix}$.
		
		\bigskip
		
		Next, consider $\bar{\bo{A}} = \bo{T}^{-1}\bo{A}\bo{T}$:
		%
		\begin{align}
			 \bo{T}\bar{\bo{A}} = \bo{A}\bo{T}
			 \\
			 \begin{bmatrix}
				\alpha \bo{B} & (\bo{P} + \bo{B}\beta\T)
			 \end{bmatrix}
			 = 
			 \bo{A}
			 \begin{bmatrix}
			 	\bo{P}& \bo{B}
			 \end{bmatrix}
		\end{align}
		
		
		This is the solution to the posed problem. Note that we either \emph{choose a new output} $\bar{\bo{C}} = \bo{e}_1\T = [1 \ 0 \ ... \ 0]$ or original output is such that $\bo{C}\bo{T} = \bo{e}_1\T$.
		
		
	\end{flushleft}
\end{frame}





\myqrframe


\end{document}
