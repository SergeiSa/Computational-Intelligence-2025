\documentclass{beamer}

\pdfmapfile{+sansmathaccent.map}


\mode<presentation>
{
	\usetheme{Warsaw} % or try Darmstadt, Madrid, Warsaw, Rochester, CambridgeUS, ...
	\usecolortheme{seahorse} % or try seahorse, beaver, crane, wolverine, ...
	\usefonttheme{serif}  % or try serif, structurebold, ...
	\setbeamertemplate{navigation symbols}{}
	\setbeamertemplate{caption}[numbered]
} 


%%%%%%%%%%%%%%%%%%%%%%%%%%%%
% itemize settings

\definecolor{mypaleblue}{RGB}{240, 240, 255}
\definecolor{mylightblue}{RGB}{120, 150, 255}
\definecolor{myblue}{RGB}{90, 90, 255}
\definecolor{mygblue}{RGB}{70, 110, 240}
\definecolor{mydarkblue}{RGB}{0, 0, 180}
\definecolor{myblackblue}{RGB}{40, 40, 120}

\definecolor{mygreen}{RGB}{0, 200, 0}
\definecolor{mydarkgreen}{RGB}{0, 120, 0}
\definecolor{mygreen2}{RGB}{245, 255, 230}

\definecolor{mygray}{gray}{0.8}
\definecolor{mygray2}{RGB}{130, 130, 130}
\definecolor{mydarkgray}{RGB}{80, 80, 160}
\definecolor{mylightgray}{RGB}{160, 160, 160}

\definecolor{mydarkred}{RGB}{160, 30, 30}
\definecolor{mylightred}{RGB}{255, 150, 150}
\definecolor{myred}{RGB}{200, 110, 110}
\definecolor{myblackred}{RGB}{120, 40, 40}

\definecolor{mypink}{RGB}{255, 30, 80}
\definecolor{myhotpink}{RGB}{255, 80, 200}
\definecolor{mywarmpink}{RGB}{255, 60, 160}
\definecolor{mylightpink}{RGB}{255, 80, 200}
\definecolor{mydarkpink}{RGB}{155, 25, 60}

\definecolor{mydarkcolor}{RGB}{60, 25, 155}
\definecolor{mylightcolor}{RGB}{130, 180, 250}

\setbeamertemplate{itemize items}[default]

\setbeamertemplate{itemize item}{\color{myblackblue}$\blacksquare$}
\setbeamertemplate{itemize subitem}{\color{mydarkblue}$\blacktriangleright$}
\setbeamertemplate{itemize subsubitem}{\color{mygray}$\blacksquare$}

\setbeamercolor{palette quaternary}{fg=white,bg=mygblue} %mydarkgray
\setbeamercolor{titlelike}{parent=palette quaternary}

\setbeamercolor{palette quaternary2}{fg=white,bg=mygblue}%black myblue
\setbeamercolor{frametitle}{parent=palette quaternary2}

\setbeamerfont{frametitle}{size=\Large,series=\scshape}
\setbeamerfont{framesubtitle}{size=\normalsize,series=\upshape}


%%%%%%%%%%%%%%%%%%%%%%%%%%%%
% block settings

%\setbeamercolor{block title}{bg=red!50,fg=black}
%\setbeamercolor{block title}{bg=mylightblue,fg=black}
\setbeamercolor{block title}{bg=myblackblue,fg=white}

\setbeamercolor*{block title example}{bg=mygreen!40!white,fg=black}

\setbeamercolor*{block body example}{fg= black,
	bg= mygreen2}


%%%%%%%%%%%%%%%%%%%%%%%%%%%%
% URL settings
\hypersetup{
	colorlinks=false,
	linkcolor=blue,
	filecolor=blue,      
	urlcolor=blue,
}

%%%%%%%%%%%%%%%%%%%%%%%%%%

\renewcommand{\familydefault}{\rmdefault}

\usepackage{amsmath}
\usepackage{mathtools}

\usepackage{subcaption}

\usepackage{qrcode}

\newcommand{\bo}[1] {\mathbf{#1}}
\newcommand{\R}{\mathbb{R}} 
\newcommand{\T}{^\top}     



\newcommand{\mydate}{Spring 2025}

\newcommand{\mygit}{\textcolor{blue}{\href{https://github.com/SergeiSa/Computational-Intelligence-2025}{github.com/SergeiSa/Computational-Intelligence-2025}}}

\newcommand{\myqr}{ \textcolor{black}{\qrcode[height=1.5in]{https://github.com/SergeiSa/Computational-Intelligence-2025}}
}

\newcommand{\myqrframe}{
	\begin{frame}
		\centerline{Lecture slides are available via Github, links are on Moodle:}
		\bigskip
		\centerline{\mygit}
		\bigskip
		\myqr
	\end{frame}
}


\newcommand{\bref}[2] {\textcolor{blue}{\href{#1}{#2}}}



%%%%%%%%%%%%%%%%%%%%%%%%%%%%
% code settings

\usepackage{listings}
\usepackage{color}
% \definecolor{mygreen}{rgb}{0,0.6,0}
% \definecolor{mygray}{rgb}{0.5,0.5,0.5}
\definecolor{mymauve}{rgb}{0.58,0,0.82}
\lstset{ 
	backgroundcolor=\color{white},   % choose the background color; you must add \usepackage{color} or \usepackage{xcolor}; should come as last argument
	basicstyle=\footnotesize,        % the size of the fonts that are used for the code
	breakatwhitespace=false,         % sets if automatic breaks should only happen at whitespace
	breaklines=true,                 % sets automatic line breaking
	captionpos=b,                    % sets the caption-position to bottom
	commentstyle=\color{mygreen},    % comment style
	deletekeywords={...},            % if you want to delete keywords from the given language
	escapeinside={\%*}{*)},          % if you want to add LaTeX within your code
	extendedchars=true,              % lets you use non-ASCII characters; for 8-bits encodings only, does not work with UTF-8
	firstnumber=0000,                % start line enumeration with line 0000
	frame=single,	                   % adds a frame around the code
	keepspaces=true,                 % keeps spaces in text, useful for keeping indentation of code (possibly needs columns=flexible)
	keywordstyle=\color{blue},       % keyword style
	language=Octave,                 % the language of the code
	morekeywords={*,...},            % if you want to add more keywords to the set
	numbers=left,                    % where to put the line-numbers; possible values are (none, left, right)
	numbersep=5pt,                   % how far the line-numbers are from the code
	numberstyle=\tiny\color{mygray}, % the style that is used for the line-numbers
	rulecolor=\color{black},         % if not set, the frame-color may be changed on line-breaks within not-black text (e.g. comments (green here))
	showspaces=false,                % show spaces everywhere adding particular underscores; it overrides 'showstringspaces'
	showstringspaces=false,          % underline spaces within strings only
	showtabs=false,                  % show tabs within strings adding particular underscores
	stepnumber=2,                    % the step between two line-numbers. If it's 1, each line will be numbered
	stringstyle=\color{mymauve},     % string literal style
	tabsize=2,	                   % sets default tabsize to 2 spaces
	title=\lstname                   % show the filename of files included with \lstinputlisting; also try caption instead of title
}

%%%%%%%%%%%%%%%%%%%%%%%%%%%%
% tikz settings

\usepackage{tikz}
\tikzset{every picture/.style={line width=0.75pt}}

%%%%%%%%%%%%%%%%%%%%%%%%%%%%




\title{Introduction}
\subtitle{Computational Intelligence, Lecture 1}
\author{by Sergei Savin}
\centering
\date{\mydate}



\begin{document}
\maketitle


\begin{frame}{Content}

\begin{itemize}
\item Motivation
\item Motivating example
\begin{itemize}
\item fmincon
\item quadprog
\item SVD-based solution
\item CVX-based solution
\end{itemize}
\item Homework
\end{itemize}

\end{frame}



\begin{frame}{Motivation}
% \framesubtitle{Parameter estimation}
\begin{flushleft}


A lot of modern methods, computations and algorithms are backed by numerical optimization tools. In this course we will study numerical optimization with an emphasis on convex methods.

\begin{block}{What we want?}
  To go from "I hope it works" to a solid understanding of the mathematics and use-cases of those tools.
\end{block}

\begin{block}{Why we want it?}
  It should allow us to solve a much wider range of problems, and solve them more effectively.
\end{block}

\end{flushleft}
\end{frame}


\begin{frame}{Motivating example}
% \framesubtitle{Parameter estimation}
\begin{flushleft}

We have the following problem: find such $\mathbf x$ that minimizes $\mathbf x^\top \mathbf M \mathbf x$, while $\mathbf C \mathbf x = \mathbf y$. In other words:

\begin{equation} \label{eq:Example1:1}
\begin{aligned}
& \underset{\mathbf  x}{\text{minimize}}
& & \mathbf x^\top \mathbf M \mathbf x, \\
& \text{subject to}
& & \mathbf C \mathbf x = \mathbf y.
\end{aligned}
\end{equation}

More concrete:

\begin{equation} \label{eq:Example1:2}
\begin{aligned}
& \underset{\mathbf  x}{\text{minimize}}
& & \begin{bmatrix} x_1 & x_2 & x_3 \end{bmatrix}
\begin{bmatrix}  
   1 & 0 & 1 \\
   0 & 5 & 0 \\
   1 & 0 & 3 \\
   \end{bmatrix}
\begin{bmatrix} x_1 \\ x_2 \\ x_3 \end{bmatrix}, \\
& \text{subject to}
& & \begin{bmatrix} 1 & 7 & 2 \end{bmatrix}
\begin{bmatrix} x_1 \\ x_2 \\ x_3 \end{bmatrix}
= 1.
\end{aligned}
\end{equation}

How do we solve it?

\end{flushleft}
\end{frame}



\begin{frame}[fragile]{Motivating example}
\framesubtitle{fmincon}
\begin{flushleft}

One very popular way of doing it is by use of a general-purpose local optimization solver, such as \texttt{fmincon} provided by MATLAB. Here is one possible solution:

\begin{lstlisting}[language=Matlab]
M = [1 0 1; 0 5 0; 1 0 3]; 
C = [1 7 2]; 
y = 1;

fnc = @(x) x'*M*x;
con = @(x) deal([], C*x-y);
x = fmincon(fnc, zeros(3, 1), [],[],[],[],[],[], con)
\end{lstlisting}

Average solution time is \textbf{4.8} ms (this depends on many factors, so treat it only as a relative information). 
Solution is $\mathbf{x} = \begin{bmatrix} 0.0442 & 0.1239 & 0.0442 \end{bmatrix}$.

\end{flushleft}
\end{frame}


\begin{frame}[fragile]{Motivating example}
\framesubtitle{quadprog}
\begin{flushleft}

A more sophisticated, but still a very straightforward approach is to use a dedicated solver for this class of problems \texttt{quadprog} provided by MATLAB. Here is the solution:

\begin{lstlisting}[language=Matlab]
M = [1 0 1; 0 5 0; 1 0 3]; 
C = [1 7 2]; 
y = 1;

x = quadprog(M,[],[],[],C,y)
\end{lstlisting}

Average solution time is \textbf{0.56} ms, an order of magnitude less than with \texttt{fmincon}.

\end{flushleft}
\end{frame}



\begin{frame}[fragile]{Motivating example}
\framesubtitle{SVD-based solution}
\begin{flushleft}

We can use an algebraic solution, based on SVD decomposition (or its derivative methods - null space and pseudo-inverse), as follows:

\begin{lstlisting}[language=Matlab]
M = [1 0 1; 0 5 0; 1 0 3]; 
C = [1 7 2]; 
y = 1;

tol = 10^(-5);
[P, N] = pinv_null(C, tol);
x = ( eye(3) - N*((N'*M*N) \ (N'*M)) ) * P*y
\end{lstlisting}

Where \texttt{pinv\_null} is a function combining \texttt{pinv} and \texttt{null}, obtained from a single SVD decomposition.

Average solution time is \textbf{0.027} ms, $\sim 20$ times faster than \texttt{quadprog} and $\sim 200$ times faster than \texttt{fmincon}.

\end{flushleft}
\end{frame}



\begin{frame}[fragile]{Motivating example}
\framesubtitle{CVX-based solution}
\begin{flushleft}

Finally, we can invoke one of the most powerful convex optimization tools with a user-friendly coding style - CVX:

\begin{lstlisting}[language=Matlab]
M = [1 0 1; 0 5 0; 1 0 3];
C = [1 7 2];
y = 1;

cvx_begin
variables x(3);
minimize( x' * M * x );
subject to
    C*x == y;
cvx_end
\end{lstlisting}

However, we will see that the overhead for the call to the solver for this task is excessive. Average solution time is \textbf{282} ms, which is $\sim 60$ times slower than \texttt{fmincon}.

\end{flushleft}
\end{frame}




%\begin{frame}{Homework}
%% \framesubtitle{Parameter estimation}
%\begin{flushleft}
%
%Solve problem \eqref{eq:Example1:2} by the method you know and understand.
%
%\end{flushleft}
%\end{frame}

\myqrframe

\end{document}
